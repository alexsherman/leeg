\documentclass[12pt,letterpaper]{article}
	% basic article document class
	% use percent signs to make comments to yourself -- they will not show up.

\usepackage{mathtools}
\usepackage{amssymb}
	% packages that allow mathematical formatting
	
\usepackage{amsmath}
\DeclareMathOperator*{\argmax}{argmax}
\DeclareMathOperator*{\argmin}{argmin}
    % more math and argmin, argmax
    
\usepackage{color}
\usepackage{xcolor}
\usepackage{graphicx}
	% package that allows you to include graphics
    
\usepackage{titling}
    % package for convenient titling
    
\usepackage[english]{babel}
\usepackage[utf8]{inputenc}
\usepackage[top=1in, bottom=1in, left=1in, right=1in]{geometry}
\usepackage{amsthm}
	% for theorems and proofs

\frenchspacing
	% one space after periods
\usepackage{fancyhdr}
	% allows custom headers
	
\usepackage{listings}
    % for code-style text within newcommand arguments

\pagestyle{fancy}

\renewcommand{\footrulewidth}{0.4pt}
	%footer

\title{Champion Recommendation Models}
\author{Daniel McFalls}
\date{May 4, 2019}

\setlength{\headheight}{28pt}
\lhead[]{\thetitle}
\rhead{\theauthor}

\cfoot{\thepage}

\newtheorem{theorem}{Theorem}[section]
\newtheorem{corollary}{Corollary}[section]
\newtheorem{lemma}{Lemma}[section]

\newcommand\TODO[1]{\textcolor{cyan}{\lstinline{TODO: #1}}}

\renewcommand\qedsymbol{$Q.E.D.$}

\setlength{\parindent}{0pt}

\begin{document}
\thispagestyle{fancy} % shows header/footer

\maketitle

\textit{This paper on League of Legends presents various formal data-driven
models for recommending a champion to pick during champion-select based on a
player's match history and the existing picks on the two teams.}

\begin{section}{Preliminaries}

Let us define the following:

\begin{itemize}
    \item Let $N$ be an ordered collection of all champions in League of
    Legends.
    \item An $N_i$ denotes the champion at index $i$ in $N$
    \item Define the number of champions as $n=|N|$
\end{itemize}

Assume the following data is available for some summoner $S$:

\begin{itemize}
    \item Let $M$ be a vector containing the match history for $S$
    \item Let each $M_i$ be a game $G$
    \item Each $G$ contains a list of five champions $G[S]$ including a champion
    played by $S$, and a list of five champions $G[O]$ on the opposing team
    \item Each $G$ also contains $G_w$, defined to be \verb|true| if $S$ won the
    match, and \verb|false| otherwise
\end{itemize}

Now we define the following:

\begin{itemize}
    \item Let $F$ be an $n \times n$ matrix where $F_{ij}$ denotes the number of
    games $S$ has played as champion $N_i$ against champion $N_j$
    \item Let $T$ be an $n \times n$ matrix where $T_{ij}$ denotes the number of
    games $S$ has played as champion $N_i$ on a team with $N_j$
    \item Let $W$ be an $n \times n$ matrix where $W_{ij}$ denotes the number of
    wins that $S$ has on champion $N_i$ against champion $N_j$
    \item Let $V$ be an $n \times n$ matrix where $V_{ij}$ denotes the number of
    wins that $S$ has on champion $N_i$ on a team with $N_j$
\end{itemize}

\end{section}

\begin{section}{Win-rate Models}

Here we propose a few models of varying complexity for answering the question,
"If I'm picking my champion and some champions are already picked to be on my
team and on the opposing team, what should I pick to ensure my highest chances
of victory?"

\subsection{Simple Independent Win-rates}

A simple intuitive approach to recommending a pick at an arbitrary point along
the pick-ban phase is to take the pick that synergizes best with existing team
mates' picks and performs best against the opposing team's picks.
\hfill \vspace{2mm}

We assume, for this first model, that win-rates between champion pairs can be
treated independently of which other champions appeared in the games.
\hfill \vspace{2mm}

Furthermore, this first model will only consider relationships between pairs of
champions.
\hfill \vspace{2mm}

Finally, we incorporate only data from the match history of $S$ and how $S$ has
performed with and against other champions on a given champion. This restricts
recommendations to champions that $S$ has played.
\hfill \vspace{2mm}

Now, we define a $\verb|Score|_i$ for $S$ on champion $N_i$ as follows:
\begin{equation}
    \verb|Score|_i =
    \prod_{j \in {G[S]}} { \frac { V_{ij} } { T_{ij} } } \times
    \prod_{j \in {G[O]}} { \frac { W_{ij} } { F_{ij} } }
\end{equation}

This number is simply the product for $S$ of win-rates with champions on the
allied team and of win-rates against champions on the opposing team.

Now it is easy to define a $\verb|Recommendation|$ as follows
\begin{equation}
    \verb|Recommendation| = 
    \argmax_i({\verb|Score|_i})
\end{equation}

Any $k$ recommendations can be generated by taking each of the $k^{\verb|th|}$
order statistics from the collection of available \verb|Score|s.

\subsection{Independent Win-rates}

Interactions between champions extend beyond 1:1 relationships. For example, it
would be nice if our model could take into account synergy with Malphite
\textit{and} Orianna or advantage against Xayah \textit{and} Rakan.
\hfill \vspace{2mm}

A simple way to accomplish this is to populate additional structures $V$, $T$,
$W$, and $F$ which is queried by, instead of a pair of champion-indices, a
set of champion-indices.
\hfill \vspace{2mm}

For example, consider generalized $V$ whose superscript denotes $|C|-1$.
We define $V^{3}$ to be an $n \times n \times n$ matrix where each $V^{3}_{iC}$
denotes the number of wins that $S$ has on champion $N_i$ playing on a team with
the champions in $C$.
\hfill \vspace{2mm}

For $V$ we can now populate the generalized structures $V^2$ up to $V^5$. These
structures are space-intensive, but not prohibitively so. They can all be
generated efficiently by iterating through the match history of $S$.
\hfill \vspace{2mm}

It is now possible to extend our model for $\verb|Score|_{i}^{k}$ to include the
generalized win-rate structures defined above:

\begin{equation}
    \verb|Score|_{i}^{k} =
    \prod_{|C| = k, C \in {G[S]}} { \frac { V_{iC} } { T_{iC} } } \times
    \prod_{|C| = k, C \in {G[O]}} { \frac { W_{iC} } { F_{iC} } }
\end{equation}

The adjusted formula for $\verb|Recommendation|_k$ follows:

\begin{equation}
    \verb|Recommendation|_k = 
    \argmax_i({\verb|Score|_{i}^{k}})
\end{equation}

\end{section}

\begin{section}{Complex Models}

\TODO{develop more complex models and add additional sections describing them}

\end{section}

\begin{section}{Analysis}

\TODO{analyze time-complexity and memory-complexity for the models discussed}

\end{section}

\newpage

\end{document}